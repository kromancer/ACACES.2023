\documentclass{acaces}

\usepackage{minted}

\begin{document}

\title{DPREF: Using Dead Code Elimination for Prefetching Irregular Memory Accesses}

\author{
Konstantinos~Sotiropoulos\addressnum{1},\extranum{1},
Jonas~Skeppstedt\addressnum{2},\extranum{2},
Angelos~Arelakis\addressnum{3},\extranum{3},
Per~Stenström\addressnum{1},\extranum{1}
}

\address{1}{
  Chalmers University of Technology
}

\address{2}{
  Lund University
}

\address{3}{
  ZeroPoint Technologies
}

\extra{1}{\{konstantinos.sotirpoulos,per.stenstrom\}@chalmers.se}
\extra{2}{jonas.skeppstedt@cs.lth.se}
\extra{3}{angelos.arelakis@zptcorp.com}

\pagestyle{empty}

\begin{abstract}
  A single paragraph, distilled from the introduction.
  Mention that this is a work in progress.
\end{abstract}

\keywords{irregular applications; prefetching; pipelined parallelism; DCE}

\section{Introduction}

Irregular memory accesses pose a significant challenge in applications within graph analytics, database management systems, and sparse linear algebra.
These applications experience high execution times due to frequent stalls caused by long latency DRAM accesses.
Prefetching is a latency avoidance technique that aims to bring useful data closer to the processor before it is requested.

Processors are equipped with dedicated hardware prefetchers that predict future accesses and do so transparently, without requiring any software intervention.
These prefetchers are efficient for applications with regular access patterns, such as iterating over a dense matrix.
Nonetheless, they struggle with indirect memory accesses, like those encountered when traversing a sparse graph.

There have been several proposals to improve the predictive capabilities of hardware prefetchers e.g. with the application of machine learning \cite{bera_pythia_2021}.
However, due to their speculative nature, these techniques can not be accurate, fetching irrelevant alongside useful data, and thus wasting bandwidth, on-chip space and energy.

To enhance accuracy, another class of proposals aims to add some programmability to hardware prefetchers \cite{basak_analysis_2019, talati_prodigy_2021, yang_spzip_2021}.
Still, this class of prefetchers has limited applicability, as their constrained programming model is only able to cover certain data structures and representations.
Moreover, programming these prefetchers necessitates the use of a different programming language from the one used to express the application,
creating a learning barrier that hinders the wider adoption of such proposals.

In addition to hardware prefetchers, processor ISAs include specialized instructions to explicitly request data prefetches
Programmers and compilers can leverage their knowledge of the underlying microarchitecture and an application's access patterns to utilize these instructions.
This technique, known as software prefetching, has a fundamental limitation:
the timing of prefetches, as expressed staticaly by a look-ahead distance or the location of a prefetch instruction, is hard to reason about.
This limitation becomes particularly evident with irregular applications,
where software prefetching techniques provide modest performance gains \cite{tran_clairvoyance_2017, ainsworth_software_2018}.

A recent wave of proposals aims to provide timely and accurate prefetches for irregular applications that use sparse data structures,
demonstrating very promising results \cite{manocha_graphattack_2021, nguyen_phloem_2023}.
Due to the way sparse data structures are represented,
their traversal exhibits a specific type of irregularity, indirection, typically taking the form \mintinline{C}{A[B[C[i]]]}.
By decoupling such multi-level indirection into distinct pipeline stages, each stage housing a slice of the application,
this approach exposes more opportunities for prefetching.

These proposals represent examples of hardware-software co-design:
the transformation is enacted by the compiler, and they offer ISA-visible queues for stage-to-stage communication.
These queues benefit from low-latency hardware implementation as inter-stage communication occurs every few cycles.

The primary aim of this study is to extend these proposals with the following contributions:
\begin{itemize}
  
\item We propose repurposing a well-established compiler optimization technique, Dead Code Elimination (DCE) \cite{cytron_efficiently_1991}, to perform the application partitioning.
      For generating multiple slices or stages, the partitioning process can be applied recursively.

\item We suggest transforming the application into an arbitrary dataflow network, not merely feed-forward pipelines.
      This approach would accommodate any dependencies among the slices.

\item To enable the compiler to analyze such transformations, we propose modeling the resulting dataflow network as a Petri net.

\end{itemize}

\section{Background}

With DCE, the compiler partitions an application into redundant and essential code, and then discards the redundant portion.
The algorithm maintains a workset of live instructions.
The workset is initialized with a set of pre-live instructions, such as memory stores, that explicitly modify system state.
Live instructions are iteratively removed from the workset for control- and data-dependency examination,
subsequently adding more instructions based on outcome of this examination.

With an intermediate representation of a program in static single-assingment form,
identifying data dependencies is straightforward due to the unambiguous definition of each variable.
Control dependencies, on the other hand, necessitate a post-dominance analysis on the CFG.
A live instruction is deemed control-dependent on a conditional statement if the basic block containing the live instruction is part of the post-dominance frontier of the basic block that houses the conditional. 
In other words, if the live instruction is not always on the path from a conditional to the program's exit, then the conditional should be considered live.

DCE can become a versatile slicing technique given the flexibility in redefining the initialization of the workset.
For our approach, we harness the programmer's insights into the application and invite them to annotate a specific memory access.
This annotated memory access serves as the initial point of severance, effectively jump-starting the slicing process.

\section{DPREF on BFS} 

2 pages

Demonstrate:
\begin{itemize}
\item Our approach on BFS
\item How the user guides the compiler
\item The Loss of Decoupling (LoD) \cite{bird_effectiveness_1993} from the RAW on the \mintinline{C}{Frontier}
\item The recursive application of DCE
\item How a Petri net can model slices, queues and MLP.
\item What kind of insights the compiler gets from the Petri net, as it dives into the recursive application of the slicing.
\item How the compiler terminates the recursion given the feedback from the Petri net.
\end{itemize}

\section{Preliminary conclusion}

1/2 page

Answer these questions:
\begin{itemize}
\item Can you state a preliminary conclusion?
\item What is next, which challenges do you need to address?
\end{itemize}


\section{Acknowledgement}

This work was partially supported by the Wallenberg AI, Autonomous Systems and Software Program (WASP)
funded by the Knut and Alice Wallenberg Foundation

\bibliography{references}

\end{document}

