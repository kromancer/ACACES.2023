\documentclass{acaces}

\begin{document}

\title{Using DCE for slicing applications with irregular memory accesses}

\pagestyle{empty}

\begin{abstract}

\end{abstract}

\keywords{irregular applications; prefetching; DCE}

\section{Introduction}

(1/2 page)

\begin{enumerate}
\item setting the scene
\item state-of-the-art
\item research problem
\item my approach in solving this problem
\item scientific contribution of the paper
\end{enumerate}

Irregular memory accesses pose a significant challenge in applications within graph analytics, database management systems, and sparse linear algebra.
These applications often experience high execution times due to frequent stalls caused by long latency DRAM accesses.
A memory access pattern is considered irregular if the processor is unable to predict and \textbf{prefetch} it effectively.

Prefetching is a latency avoidance technique that aims to bring useful data closer to the processor before it is requested.
Modern processors are equipped with dedicated \textbf{hardware} prefetchers that operate transparently, without requiring any software intervention.
These prefetchers are efficient for applications with regular access patterns, such as traversing a dense matrix.
However, they struggle with indirect memory accesses, like those encountered when traversing a linked list.

In addition to dedicated hardware, processors also provide specialized prefetching instructions.
\textbf{Software} engineers can utilize these instructions to explicitly request data prefetches, leveraging their knowledge of the application and its data structures.
Nonetheless, this approach requires significant effort and a thorough understanding of the processor's microarchitecture.
Moreover, there are currently no compiler optimizations that automatically insert prefetch instructions.

Recently, the research community has proposed several prefetching solutions specifically designed for irregular workloads.
The application of machine learning techniques can improve the predictive capabilities of hardware prefetchers when handling irregular memory access patterns.
However, this improvement comes at the cost of added hardware complexity.
Furthermore, due to their speculative nature, these techniques may not be fully \textbf{accurate}, fetching irrelevant data alongside useful information,
leading to wasted resources such as energy.

To enhance accuracy, another class of proposals aims to add some programmability to hardware prefetchers.
However, this class of prefetchers has limited \textbf{applicability},
as their constrained programming model is only able to cover certain representations of sparse data structures.
Moreover, programming these prefetchers necessitates the use of a different programming language from the one used to express the application,
creating a learning barrier that hinders the wider adoption of such proposals.

The primary \textbf{goal} of this study is to propose an accurate prefetching scheme with broader applicability.
Rather than relying on additional prefetching hardware, our solution aims to increase the utilization of existing idle resources.
The most important of these resources is the capacity of modern processors for Memory Level Parallelism (MLP),
which refers to their ability to have multiple concurrent memory requests "in-flight".

At the core of our implementation is a transformation of the application into a pipeline.
This transformation is made feasible by repurposing a well-established compiler optimization technique called Dead Code Elimination (DCE).
In traditional \textbf{DCE}, the compiler slices an application into dead (redundant) and live (essential) code, then discards the redundant portion.
In our approach, the compiler uses the re-purposed DCE to slice the application into memory and compute slices.
The memory slices contain the essential code needed to formulate the irregular memory accesses.
These memory slices are placed in the early stages of the pipeline, designed to maximize MLP as soon as possible.

Our proposal is designed to be user-friendly, as the burden of transformation is handled by the compiler, with only minimal code annotations required from the programmer.
Moreover, the accuracy and applicability of our technique are assured, as the memory slices are derived directly from the application itself,
ensuring that prefetching matches the actual memory access pattern of the application..

\section{Background}

1/2 page

\section{The "new stuff"} 

Hand slicing of the 3 apps I have looked

\section{Preliminary conclusion}

1/2 page

\bibliography{guide}

\end{document}

