\begin{lstlisting}[style=cppcode]
void secondSlice(const double *__restrict X, double *__restrict Y,
                 dpref::Queue<tuple<double, size_t>> *Q) {
\end{lstlisting}\vspace{-\baselineskip}

\begin{lstlisting}[style=cppcode, backgroundcolor=\color{yellow!15}, firstnumber=last]
  for (size_t I = 0; I < N; I++) {
\end{lstlisting}\vspace{-\baselineskip}

\begin{lstlisting}[style=cppcode, backgroundcolor=\color{green!15}, firstnumber=last]
    Y[I] = 0;
\end{lstlisting}\vspace{-\baselineskip}

\begin{lstlisting}[style=cppcode, firstnumber=last]
    for (;;) {
      auto Token = Q->pop();
      if (Token.isTerminationToken()) return;
      if (Token.isLoopTerminationToken()) break;
      auto [MVal, ColIdx] = Token.getData();
\end{lstlisting}\vspace{-\baselineskip}

\begin{lstlisting}[style=cppcode, backgroundcolor=\color{green!15}, firstnumber=last]
      Y[I] += MVal * X[ColIdx];
\end{lstlisting}\vspace{-\baselineskip}

\begin{lstlisting}[style=cppcode, firstnumber=last]
    }
\end{lstlisting}\vspace{-\baselineskip}

\begin{lstlisting}[style=cppcode, backgroundcolor=\color{yellow!15}, firstnumber=last]
  }
\end{lstlisting}\vspace{-\baselineskip}

\begin{lstlisting}[style=cppcode, firstnumber=last]
}
\end{lstlisting}\vspace{-\baselineskip}
