
\documentclass{acaces}

\begin{document}


\title{ACACES\\
Using the style file
}

\author{
Michiel~Ronsse\addressnum{1},
Other~Author\addressnum{1}\comma\extranum{1},
Yet~Another~Author\addressnum{2},
The~Last~Author\addressnum{1},\extranum{2}
}

\address{1}{
ELIS,
Ghent University,
Sint-Pietersnieuwstraat 41,
9000 Gent,
Belgium
}

\address{2}{
ELIS,
UGent,
9000 Gent,
Belgium
}

\extra{1}{E-mail: \{ronsse,oauthor,yaauthor,tlauthor\}@elis.UGent.be}
\extra{2}{This author wishes to thank nobody.}

\pagestyle{empty}


\begin{abstract}
This document describes the usage of the \LaTeX{} style provided
for the ACACES poster session.
\end{abstract}

\keywords{ACACES; poster session; \LaTeX}

\section{Introduction}
This document ({\tt guide.pdf}) describes {\tt acaces.cls}, the
\LaTeX~\cite{latex} style file to be used for the abstract of your 
\href{http://www.hipeac.net/acaces2006/}{ACACES} poster. The document
you are reading right now was also produced with that style file.
You are strongly advised to use \LaTeX{} for preparing your camera-ready paper.
If this is impossible, please try to mimic the layout of this document as
closely as possible. Word users can use the {\tt
ACACEStemplate.doc} template.

The remainder of this document assumes you use \LaTeX{} with the
{\tt acaces.cls} style file.

Using the style file is pretty straightforward, just have a look to the source of this file 
({\tt guide.tex}).


\section{Typesetter}

Use {\tt pdflatex guide.tex} to generate a PDF file, don't use {\tt latex} to
produce a dvi file and then {\tt dvipdf} or {\tt dvips} followed by {\tt
ps2pdf} or {\tt pdftopdf} as this results in ugly looking PDF.

\section{Figures} 
Your figures should also be made in PDF. Unfortunately, not a
lot of drawing applications allow you to write a PDF file. You can however
create EPS files and transform the resulting files to PDF using {\tt epstopdf
figure.eps}. Figure~\ref{logo} shows an example of a figure.
Please bear in mind that your paper will be reduced to 70\% of its
original size.


\begin{figure}
\centering
\includegraphics{hipeac-logo-bw}
\caption{An example of a Figure: the HiPEAC logo.}
\label{logo}
\end{figure}


\section{Fonts}
Please make sure that your PDF file only contains Type1 fonts. This should be
no problem if you use pdflatex and the {\tt acaces.cls} style file. If you use
another typesetting system, please check if the resulting PDF file only uses
Type1 fonts.  You can check this using Acrobat Reader: open your file and
check all fonts using File$\rightarrow$Document Properties$\rightarrow$Fonts\ldots

\section{Final submission}
Make sure that your PDF file does not contain page numbers, is
up to 4 pages long and is formatted for an A4 page.
Upload the final version before June 8 on the website.

\section{About this style}
You are encouraged to send bug reports, remarks, \ldots about this style to
\href{mailto:ronsse@elis.UGent.be}{ronsse@elis.UGent.be}.

\section{End}
This is the end! \label{end}

\bibliography{guide}

\end{document}

